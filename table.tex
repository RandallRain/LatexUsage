% -----------------------------------------------------------------
% @file table.tex
% @brief Some simple examples about how to typeset tables in LaTeX
% effectively.
% @date 2025-04-17
% @author LHT
% -----------------------------------------------------------------

\documentclass{article}
\usepackage{array}
\usepackage{lipsum} % For dummy text.

% Create new column types.
% Set column width, horizontally centered and vertically top-aligned.
\newcolumntype{P}[1]{>{\centering\arraybackslash}p{#1}}
% Set column width, horizontally and vertically centered.
\newcolumntype{M}[1]{>{\centering\arraybackslash}m{#1}}

% Set column width using a fraction of \textwidth, horizontally 
% centered and vertically top-aligned.
\newcolumntype{Q}[1]{>{\centering\arraybackslash}p{#1\textwidth}}
% Set column width using a fraction of \textwidth, horizontally and vertically
% centered.
\newcolumntype{N}[1]{>{\centering\arraybackslash}m{#1\textwidth}}

\begin{document}

\section{General Tables}

% Show the new column type P, horizontally centered and vertically top-aligned.
\begin{table}
	\centering
	\caption{New Column Type P}
	\begin{tabular}{|P{2cm}|P{2cm}|P{5cm}|}
		\hline
		\textbf{Column 1} & \textbf{Column 2} & \textbf{Column 3} \\
		\hline
		Row 1 & Cell 1,2 & \lipsum[1][1] \\ \hline
		Row 2 & Cell 2, 2 & \lipsum[2][1-2] \\ \hline
	\end{tabular}
\end{table}

% Show the new column type M, horizontally and vertically centered.
\begin{table}
	\centering
	\caption{New Column Type M}
	\begin{tabular}{|M{2cm}|M{2cm}|M{5cm}|}
		\hline
		\textbf{Column 1} & \textbf{Column 2} & \textbf{Column 3} \\
		\hline
		Row 1 & Cell 1,2 & \lipsum[1][1] \\ \hline
		Row 2 & Cell 2, 2 & \lipsum[2][1-2] \\ \hline
	\end{tabular}
\end{table}

% Show the new column type Q, horizontally centered and vertically top-aligned.
\begin{table}
	\centering
	\caption{New Column Type Q}
	\begin{tabular}{|Q{0.2}|Q{0.2}|Q{0.4}|}
		\hline
		\textbf{Column 1} & \textbf{Column 2} & \textbf{Column 3} \\
		\hline
		Row 1 & Cell 1,2 & \lipsum[1][1] \\ \hline
		Row 2 & Cell 2, 2 & \lipsum[2][1-2] \\ \hline
	\end{tabular}
\end{table}


% Show the new column type N, horizontally and vertically centered.
\begin{table}
	\centering
	\caption{New Column Type C}
	\begin{tabular}{|N{0.2}|N{0.2}|N{0.4}|}
		\hline
		\textbf{Column 1} & \textbf{Column 2} & \textbf{Column 3} \\
		\hline
		Row 1 & Cell 1,2 & \lipsum[1][1] \\ \hline
		Row 2 & Cell 2, 2 & \lipsum[2][1-2] \\ \hline
	\end{tabular}
\end{table}

Show the strange vertical alignment of the cells in a row.
\begin{table}
	\centering
	\caption{The strange vertical alignment}
	\begin{tabular}{|p{0.3\textwidth}|m{0.3\textwidth}|b{0.3\textwidth}|}
		\hline
		\textbf{Column Type p} & \textbf{Column Type m} & \textbf{Column Type b} \\
		\hline
		\lipsum[1][1-2] & \lipsum[2][1-2] & \lipsum[3][1-2] \\
		\hline
	\end{tabular}
\end{table}

\end{document}

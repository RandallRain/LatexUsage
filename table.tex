% -----------------------------------------------------------------
% @file table.tex
% @brief Some simple examples about how to typeset tables in LaTeX
% effectively.
% @date 2025-04-17
% @author LHT
% -----------------------------------------------------------------

\documentclass{article}
\usepackage{array}
\usepackage{lipsum} % for dummy text.
\usepackage{multirow} % for multirow cells. 
\usepackage{xcolor} % for color.
\usepackage{float} % for float environments.
 
% -----------------------------------------------------------------
% Create new column types for tables.
% -----------------------------------------------------------------
% Set column width, horizontally centered and vertically top-aligned.
\newcolumntype{P}[1]{>{\centering\arraybackslash}p{#1}}
% Set column width, horizontally and vertically centered.
\newcolumntype{M}[1]{>{\centering\arraybackslash}m{#1}}

% Set column width using a fraction of \textwidth, horizontally 
% centered and vertically top-aligned.
\newcolumntype{Q}[1]{>{\centering\arraybackslash}p{#1\textwidth}}
% Set column width using a fraction of \textwidth, horizontally and vertically
% centered.
\newcolumntype{N}[1]{>{\centering\arraybackslash}m{#1\textwidth}}

% -----------------------------------------------------------------
% Change the default position for float envrionments.
% H: absoulute here.
% -----------------------------------------------------------------
\makeatletter
\renewcommand{\fps@figure}{H}
\renewcommand{\fps@table}{H}
\makeatother

% -----------------------------------------------------------------
% Define the inline code command.
% -----------------------------------------------------------------
% Define a light‐gray background color
\definecolor{inlinecodebg}{RGB}{235,235,235}
% Define the inline code command.
\newcommand{\incode}[1]{%
  	\begingroup
    	\setlength{\fboxsep}{1pt}% % padding between text and box.
    	\colorbox{inlinecodebg}{\ttfamily\detokenize{#1}}%
  	\endgroup
}

% -----------------------------------------------------------------
% Other settings.
% -----------------------------------------------------------------
% Set horizontally centered for all multirow cells.
%\renewcommand\multirowsetup{\centering\arraybackslash}


\begin{document}
\section{General Tables}
Show the new column type P, horizontally centered and vertically top-aligned.
\begin{table}
	\centering
	\caption{New Column Type P}
	\begin{tabular}{|P{2cm}|P{2cm}|P{5cm}|}
		\hline
		\textbf{Column 1} & \textbf{Column 2} & \textbf{Column 3} \\
		\hline
		Row 1 & Cell 1,2 & \lipsum[1][1] \\ \hline
		Row 2 & Cell 2, 2 & \lipsum[2][1-2] \\ \hline
	\end{tabular}
\end{table}

Show the new column type M, horizontally and vertically centered.
\begin{table}
	\centering
	\caption{New Column Type M}
	\begin{tabular}{|M{2cm}|M{2cm}|M{5cm}|}
		\hline
		\textbf{Column 1} & \textbf{Column 2} & \textbf{Column 3} \\
		\hline
		Row 1 & Cell 1,2 & \lipsum[1][1] \\ \hline
		Row 2 & Cell 2, 2 & \lipsum[2][1-2] \\ \hline
	\end{tabular}
\end{table}

Show the new column type Q, horizontally centered and vertically top-aligned.
\begin{table}
	\centering
	\caption{New Column Type Q}
	\begin{tabular}{|Q{0.2}|Q{0.2}|Q{0.4}|}
		\hline
		\textbf{Column 1} & \textbf{Column 2} & \textbf{Column 3} \\
		\hline
		Row 1 & Cell 1,2 & \lipsum[1][1] \\ \hline
		Row 2 & Cell 2, 2 & \lipsum[2][1-2] \\ \hline
	\end{tabular}
\end{table}

Show the new column type N, horizontally and vertically centered.
\begin{table}
	\centering
	\caption{New Column Type C}
	\begin{tabular}{|N{0.2}|N{0.2}|N{0.4}|}
		\hline
		\textbf{Column 1} & \textbf{Column 2} & \textbf{Column 3} \\
		\hline
		Row 1 & Cell 1,2 & \lipsum[1][1] \\ \hline
		Row 2 & Cell 2, 2 & \lipsum[2][1-2] \\ \hline
	\end{tabular}
\end{table}

Show the strange vertical alignment of the cells in a row.
\begin{table}
	\centering
	\caption{The strange vertical alignment}
	\begin{tabular}{|p{0.3\textwidth}|m{0.3\textwidth}|b{0.3\textwidth}|}
		\hline
		\textbf{Column Type p} & \textbf{Column Type m} & \textbf{Column Type b} \\
		\hline
		\lipsum[1][1-2] & \lipsum[2][1-2] & \lipsum[3][1-2] \\
		\hline
	\end{tabular}
\end{table}

\section{Merge Cells}
\subsection{Merge Multi-Rows}
Use \incode{\multirow} to merge multirow cells for short text
\begin{table}
	\centering
	\caption{Merge Cells, Short Text.}
	\begin{tabular}{|m{0.2\textwidth}|m{0.2\textwidth}|m{0.4\textwidth}|}
		\hline
		\textbf{Column 1} & \textbf{Column 2} & \textbf{Column 3} \\
		\hline
		\multirow{2}{=}{Row 1 2} & Cell 1,2 & Short Text 1, 2 \\ \cline{2-3}
		& Cell 2, 2 & Short Text 2, 2\\ \hline
		Row 3 & Cell 3, 2 & Short Text 3, 3 \\ \hline
	\end{tabular}
\end{table}

Use \incode{\multirow} to merge multirow cells for long text, the original layout.
\begin{table}
	\centering
	\caption{Merge Rows, Long Text, Orignal Layout.}
	\begin{tabular}{|m{0.2\textwidth}|m{0.2\textwidth}|m{0.4\textwidth}|}
		\hline
		\textbf{Column 1} & \textbf{Column 2} & \textbf{Column 3} \\
		\hline
		\multirow{2}{=}{Row 1 2} & Cell 1,2 & \lipsum[1][1-2] \\ \cline{2-3}
		& Cell 2, 2 & \lipsum[2][1-2]\\ \hline
		Row 3 & Cell 3, 2 & \lipsum[3][1-2] \\ \hline
	\end{tabular}
\end{table}

Use \incode{nrows} to adjust the vertically centered alignment of the multirow cell.
\begin{table}
	\centering
	\caption{Merge Rows, Long Text, Using nrows.}
	\begin{tabular}{|m{0.2\textwidth}|m{0.2\textwidth}|m{0.4\textwidth}|}
		\hline
		\textbf{Column 1} & \textbf{Column 2} & \textbf{Column 3} \\
		\hline
		\multirow{5}{=}{Row 1 2} & Cell 1,2 & \lipsum[1][1-2] \\ \cline{2-3}
		& Cell 2, 2 & \lipsum[2][1-2]\\ \hline
		Row 3 & Cell 3, 2 & \lipsum[3][1-2] \\ \hline
	\end{tabular}
\end{table}

Use vmove to adjust the vertically centered alignment of the multirow cell,
using \incode{\baselineskip} as the unit.
\begin{table}
	\centering
	\caption{Merge Rows, Long Text, Using vmove.}
	\begin{tabular}{|m{0.2\textwidth}|m{0.2\textwidth}|m{0.4\textwidth}|}
		\hline
		\textbf{Column 1} & \textbf{Column 2} & \textbf{Column 3} \\
		\hline
		\multirow{2}{=}[-1.8\baselineskip]{Row 1 2} & Cell 1,2 & \lipsum[1][1-2] \\ \cline{2-3}
		& Cell 2, 2 & \lipsum[2][1-2]\\ \hline
		Row 3 & Cell 3, 2 & \lipsum[3][1-2] \\ \hline
	\end{tabular}
\end{table}

Set all multirow cells in the followling table to be horizontally centered.
\begin{table}
	\centering
	\caption{Merge Rows, Long Text, horizontally and Vertically Centered.}
	{
		% Redefine the \multirowsetup command in a group.
		\renewcommand\multirowsetup{\centering\arraybackslash}
		\begin{tabular}{|M{0.2\textwidth}|M{0.2\textwidth}|m{0.4\textwidth}|}
			\hline
			\textbf{Column 1} & \textbf{Column 2} & \textbf{Column 3} \\
			\hline
			\multirow{2}{=}[-1.8\baselineskip]{Row 1 2} & Cell 1,2 & \lipsum[1][1-2] \\ \cline{2-3}
			& Cell 2, 2 & \lipsum[2][1-2]\\ \hline
			Row 3 & Cell 3, 2 & \lipsum[3][1-2] \\ \hline
		\end{tabular}
	}
\end{table}

\end{document}
